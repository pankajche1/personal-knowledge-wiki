% Created 2022-03-20 Sun 21:03
% Intended LaTeX compiler: pdflatex
\documentclass[11pt]{article}
\usepackage[utf8]{inputenc}
\usepackage[T1]{fontenc}
\usepackage{graphicx}
\usepackage{grffile}
\usepackage{longtable}
\usepackage{wrapfig}
\usepackage{rotating}
\usepackage[normalem]{ulem}
\usepackage{amsmath}
\usepackage{textcomp}
\usepackage{amssymb}
\usepackage{capt-of}
\usepackage{hyperref}
\author{pankaj}
\date{2022-03-20}
\title{Chapter 30 : Rare Events\\\medskip
\large Rare Events subtitle}
\hypersetup{
 pdfauthor={pankaj},
 pdftitle={Chapter 30 : Rare Events},
 pdfkeywords={},
 pdfsubject={},
 pdfcreator={Emacs 27.2 (Org mode 9.4.4)}, 
 pdflang={English}}
\begin{document}

\maketitle
\tableofcontents

I visited Israel several times during a period in which suicide bombings in buses were relatively common—though of course quite rare in absolute terms. There were altogether 23 bombings between December 2001 and September 2004, which had caused a total of 236 fatalities. The number of daily bus riders in Israel was approximately 1.3 million at that time. For any traveler, the risks were tiny, but that was not how the public felt about it. People avoided buses as much as they could, and many travelers spent their time on the bus anxiously scanning their neighbors for packages or bulky clothes that might hide a bomb.

I did not have much occasion to travel on buses, as I was driving a rented car, but I was chagrined to discover that my behavior was also affected. I found that I did not like to stop next to a bus at a red light, and I drove away more quickly than usual when the light changed. I was ashamed of myself, because of course I knew better. I knew that the risk was truly negligible, and that any effect at all on my actions would assign an inordinately high “decision weight” to a minuscule probability. In fact, I was more likely to be injured in a driving accident than by stopping near a bus. But my avoidance of buses was not motivated by a rational concern for survival. What drove me was the experience of the moment: being next to a bus made me think of bombs, and these thoughts were unpleasant. I was avoiding buses because I wanted to think of something else.

My experience illustrates how terrorism works and why it is so effective: it induces an availability cascade. An extremely vivid image of death and damage, constantly reinforced by media attention and frequent conversations, becomes highly accessible, especially if it is associated with a specific situation such as the sight of a bus. The emotional arousal is associative, automatic, and uncontrolled, and it produces an impulse for protective action. System 2 may “know” that the probability is low, but this knowledge does not eliminate the self-generated discomfort and the wish to avoid it. System 1 cannot be turned off. The emotion is not only disproportionate to the probability, it is also insensitive to the exact level of probability. Suppose that two cities have been warned about the presence of suicide bombers. Residents of one city are told that two bombers are ready to strike. Residents of another city are told of a single bomber. Their risk is lower by half, but do they feel much safer?

Many stores in New York City sell lottery tickets, and business is good. The psychology of high-prize lotteries is similar to the psychology of terrorism. The thrilling possibility of winning the big prize is shared by the community and re Cmuninforced by conversations at work and at home. Buying a ticket is immediately rewarded by pleasant fantasies, just as avoiding a bus was immediately rewarded by relief from fear. In both cases, the actual probability is inconsequential; only possibility matters. The original formulation of prospect theory included the argument that “highly unlikely events are either ignored or overweighted,” but it did not specify the conditions under which one or the other will occur, nor did it propose a psychological interpretation of it. My current view of decision weights has been strongly influenced by recent research on the role of emotions and vividness in decision making. Overweighting of unlikely outcomes is rooted in System 1 features that are familiar by now. Emotion and vividness influence fluency, availability, and judgments of probability—and thus account for our excessive response to the few rare events that we do not ignore.

\section{Overestimation and Overweighting}
\label{sec:orgcf8285c}

\begin{quote}
What is your judgment of the probability that the next president of the United States will be a third-party candidate?

How much will you pay for a bet in which you receive \$1,000 if the next president of the United States is a third-party candidate, and no money otherwise?
\end{quote}


The two questions are different but obviously related. The first asks you to assess the probability of an unlikely event. The second invites you to put a decision weight on the same event, by placing a bet on it.

How do people make the judgments and how do they assign decision weights? We start from two simple answers, then qualify them. Here are the oversimplified answers:

\begin{itemize}
\item People overestimate the probabilities of unlikely events.
\item People overweight unlikely events in their decisions.
\end{itemize}

Although overestimation and overweighting are distinct phenomena, the same psychological mechanisms are involved in both: focused attention, confirmation bias, and cognitive ease.

Specific descriptions trigger the associative machinery of System 1. When you thought about the unlikely victory of a third-party candidate, your associative system worked in its usual confirmatory mode, selectively retrieving evidence, instances, and images that would make the statement true. The process was biased, but it was not an exercise in fantasy. You looked for a plausible scenario that conforms to the constraints of reality; you did not simply imagine the Fairy of the West installing a third-party president. Your judgment of probability was ultimately determined by the cognitive ease, or fluency, with which a plausible scenario came to mind.

You do not always focus on the event you are asked to estimate. If the target event is very likely, you focus on its alternative. Consider this example:

\begin{quote}
What is the probability that a baby born in your local hospital will be released within three days?
\end{quote}

You were asked to estimate the probability of the baby going home, but you almost certainly focused on the events that might cause a baby not to be released within the normal period. Our mind has a useful capability to focus spontaneously on whatever is odd, different, or unusual. You quickly realized that it is normal for babies in the United States (not all countries have the same standards) to be released within two or three days of birth, so your attention turned to the abnormal alternative. The unlikely event became focal. The availability heuristic is likely to be evoked: your judgment was probably determined by the number of scenarios of medical problems you produced and by the ease with which they came to mind. Because you were in confirmatory mode, there is a good chance that your estimate of the frequency of problems was too high.

The probability of a rare event is most likely to be overestimated when the alternative is not fully specified. My favorite example comes from a study that the psychologist Craig Fox conducted while he was Amos’s student. Fox recruited fans of professional basketball and elicited several judgments and decisions concerning the winner of the NBA playoffs. In particular, he asked them to estimate the probability that each of the eight participating teams would win the playoff; the victory of each team in turn was the focal event.

You can surely guess what happened, but the magnitude of the effect that Fox observed may surprise you. Imagine a fan who has been asked to estimate the chances that the Chicago Bulls will win the tournament. The focal event is well defined, but its alternative—one of the other seven teams winning—is diffuse and less evocative. The fan’s memory and imagination, operating in confirmatory mode, are trying to construct a victory for the Bulls. When the same person is next asked to assess the chances of the Lakers, the same selective activation will work in favor of that team. The eight best professional basketball teams in the United States are all very good, and it is possible to imagine even a relatively weak team among them emerging as champion. The result: the probability judgments generated successively for the eight teams added up to 240\%! This pattern is absurd, of course, because the sum of the chances of the eight events must add up to 100\%. The absurdity disappeared when the same judges were asked whether the winner would be from the Eastern or the Western conference. The focal event and its alternative were equally specific in that question and the judgments of their probabilities added up to 100\%.

To assess decision weights, Fox also invited the basketball fans to bet on the tournament result. They assigned a cash equivalent to each bet (a cash amount that was just as attractive as playing the bet). Winning the bet would earn a payoff of \$160. The sum of the cash equivalents for the eight individual teams was \$287. An average participant who took all eight bets would be guaranteed a loss of \$127! The participants surely knew that there were eight teams in the tournament and that the average payoff for betting on all of them could not exceed \$160, but they overweighted nonetheless. The fans not only overestimated the probability of the events they focused on—they were also much too willing to bet on them.

These findings shed new light on the planning fallacy and other manifestations of optimism. The successful execution of a plan is specific and easy to imagine when one tries to forecast the outcome of a project. In contrast, the alternative of failure is diffuse, because there are innumerable ways for things to go wrong. Entrepreneurs and the investors who evaluate their prospects are prone both to overestimate their chances and to overweight their estimates.

\section{Vivid Outcomes}
\label{sec:org83f6f56}

As we have seen, prospect theory differs from utility theory in the relationship it suggests between probability and decision weight. In utility theory, decision weights and probabilities are the same. The decision weight of a sure thing is 100, and the weight that corresponds to a 90\% chance is exactly 90, which is 9 times more than the decision weight for a 10\% chance. In prospect theory, variations of probability have less effect on decision weights. An experiment that I mentioned earlier found that the decision weight for a 90\% chance was 71.2 and the decision weight for a 10\% chance was 18.6. The ratio of the probabilities was 9.0, but the ratio of the decision weights was only 3.83, indicating insufficient sensitivity to probability in that range. In both theories, the decision weights depend only on probability, not on the outcome. Both theories predict that the decision weight for a 90\% chance is the same for winning \$100, receiving a dozen roses, or getting an electric shock. This theoretical prediction turns out to be wrong.

TODO

\section{Speaking of Rare Events}
\label{sec:orge2d9dfb}

\begin{itemize}
\item “Tsunamis are very rare even in Japan, but the image is so vivid and compelling that tourists are bound to overestimate their probability.”

\item “It’s the familiar disaster cycle. Begin by exaggeration and overweighting, then neglect sets in.”

\item “We shouldn’t focus on a single scenario, or we will overestimate its probability. Let’s set up specific alternatives and make the probabilities add up to 100\%.”

\item “They want people to be worried by the risk. That’s why they describe it as 1 death per 1,000. They’re counting on denominator neglect.”
\end{itemize}
\end{document}